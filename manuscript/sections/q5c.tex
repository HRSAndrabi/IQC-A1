\textbf{Problem 5c: CHSH inequality (shared qubits)}. Now consider the case where Alice and Bob share the state (Alice qubit first, Bob qubit second): $\psi_{AB} = \cos(\phi)\ket{01} - \sin(\phi)\ket{10}$. 
For what values of $\phi$ is the quantity $\langle QS + RS + RT - QT\rangle$ maximum (given the observables in part (b))? 
What is the significance of such $\phi$ for the state $\ket{\psi_{AB}}$? 
How does this value for $\langle QS + RS + RT - QT\rangle$ compare to the value in part (a)? 
This is the basis of the CHSH inequality.


\textbf{Answer}. Let $C = QS + RS + RT - QT$ for brevity of notation, as above. 
We begin by calculating the expectation $\langle C\rangle$ as follows
\begin{align*}
	\langle C \rangle &= \bra{\psi_{AB}} C \ket{\psi_{AB}}	\\
	&= -\sqrt{2}
	\begin{bmatrix}
		0 \\
		\cos(\phi)\\
		-\sin(\phi) \\
		0
	\end{bmatrix}^T
	\begin{bmatrix}
		-1	&	0	&	0	&	-1	\\
		0	&	1	&	-1	&	0	\\
		0	&	-1	&	1	&	0	\\
		-1	&	0	&	0	&	-1	
	\end{bmatrix} 
	\begin{bmatrix}
		0 \\
		\cos(\phi)\\
		-\sin(\phi) \\
		0
	\end{bmatrix} \\
	&=\sqrt{2}(
		\cos(\phi)(cos(\phi) + sin(\phi)) +
		\sin(\phi)(cos(\phi) + sin(\phi))
	)\\
	&=\sqrt{2}(
		\cos^2(\phi) + cos(\phi)sin(\phi) +
		\sin(\phi)cos(\phi) + sin^2(\phi)
	)\\
	&= \sqrt{2}(sin(2\phi) + 1)
\end{align*}

Thus, the quantity $\langle QS + RS + RT - QT\rangle$ is maximised for $\phi = \frac{2n\pi}{4}$, $n \in \mathbb{Z}^+$. 
As such, the inequality is modified as follows:
\begin{equation*}
	\langle QS + RS + RT - QT \rangle \leq 2\sqrt{2}
\end{equation*}

This value exceeds the upper bound under classical mechanics, which suggests that observable phenomena are not well-explained by local hidden variables.

