\textbf{Problem 5b: CHSH inequality (quantum states)}. Next we consider measuring the expectation value of the quantity $QS + RS + RT - QT$ when Alice and Bob have access to quantum states. 
What are the eigenvalues for each of $Q, R, S$, and $T$? 
What is the expectation value $\langle QS + RS + RT - QT \rangle$ for the product state $\ket{\psi_{\text{Alice}}} \otimes \ket{\psi_{\text{Bob}}}$? 
What is the max possible value?


\textbf{Answer}. We can determine the eigenvalues of Alice and Bob's observables ($Q, R, S, T$) by solving their characteristic equations.

The eigenvalues of $Q$ are given by
\begin{align*}
	\det(\lambda I - Z) &= 0 \\
	\det(\begin{bmatrix}
		\lambda - 1 & 0 \\
		0 & \lambda + 1
	\end{bmatrix}) &= 0\\
	(\lambda - 1)(\lambda + 1) &= 0 \\
	\lambda &= \pm 1
\end{align*}

The eigenvalues of $R$ are given by
\begin{align*}
	\det(\lambda I - X) &= 0 \\
	\det(\begin{bmatrix}
		\lambda & -1 \\
		-1 & \lambda
	\end{bmatrix}) &= 0\\
	\lambda^2 - 1 &= 0 \\
	\lambda &= \pm 1
\end{align*}

The eigenvalues of $S$ are given by
\begin{align*}
	\det(\lambda I - \left(-\frac{Z+X}{\sqrt{2}}\right)) &= 0 \\
	\det(\begin{bmatrix}
		\lambda + \frac{1}{\sqrt{2}} & \frac{1}{\sqrt{2}} \\
		\frac{1}{\sqrt{2}} & \lambda - \frac{1}{\sqrt{2}}
	\end{bmatrix}) &= 0\\
	(\lambda + \frac{1}{\sqrt{2}})(\lambda - \frac{1}{\sqrt{2}}) - \frac{1}{2} &= 0 \\
	\lambda &= \pm 1
\end{align*}

The eigenvalues of $T$ are given by
\begin{align*}
	\det(\lambda I - \frac{Z-X}{\sqrt{2}}) &= 0 \\
	\det(\begin{bmatrix}
		\lambda - \frac{1}{\sqrt{2}} & \frac{1}{\sqrt{2}} \\
		\frac{1}{\sqrt{2}} & \lambda + \frac{1}{\sqrt{2}}
	\end{bmatrix}) &= 0\\
	(\lambda - \frac{1}{\sqrt{2}})(\lambda + \frac{1}{\sqrt{2}}) - \frac{1}{2} &= 0 \\
	\lambda &= \pm 1
\end{align*}

We calculate the expectation value of $\langle QS + RS + RT - QT \rangle$ with reference to the state $\ket{\psi_{AB}}$ by evaluating the expression
\begin{align*}
	\langle QS + RS + RT - QT \rangle &= \bra{\psi_{AB}} QS + RS + RT - QT \ket{\psi_{AB}}	
\end{align*}

Let $C = QS + RS + RT - QT$ for brevity of notation.
Begin by calculating tensor products $QS, RS, RT$ and $QT$
\begin{align*}
	QS &= Z \otimes -\frac{Z+X}{\sqrt{2}} 
	= -\frac{1}{\sqrt{2}}\begin{bmatrix}
		1	&	1	&	0	&	0	\\
		1	&	-1	&	0	&	0	\\
		0	&	0	&	-1	&	-1	\\
		0	&	0	&	-1	&	1	
	\end{bmatrix} \\
	RS &= X \otimes -\frac{Z+X}{\sqrt{2}} 
	= -\frac{1}{\sqrt{2}}\begin{bmatrix}
		0	&	0	&	1	&	1	\\
		0	&	0	&	1	&	-1	\\
		1	&	1	&	0	&	0	\\
		1	&	-1	&	0	&	0	
	\end{bmatrix} \\
	RT &= X \otimes \frac{Z-X}{\sqrt{2}} 
	= -\frac{1}{\sqrt{2}}\begin{bmatrix}
		0	&	0	&	1	&	-1	\\
		0	&	0	&	-1	&	-1	\\
		1	&	-1	&	0	&	0	\\
		-1	&	-1	&	0	&	0	
	\end{bmatrix} \\
	QT &= Z \otimes \frac{Z-X}{\sqrt{2}} 
	= -\frac{1}{\sqrt{2}}\begin{bmatrix}
		1	&	-1	&	0	&	0	\\
		-1	&	-1	&	0	&	0	\\
		0	&	0	&	-1	&	1	\\
		0	&	0	&	1	&	1	
	\end{bmatrix} 
\end{align*}

Thus, $C$ is given by
\begin{equation*}
	C
	= -\sqrt{2}\begin{bmatrix}
		-1	&	0	&	0	&	-1	\\
		0	&	1	&	-1	&	0	\\
		0	&	-1	&	1	&	0	\\
		-1	&	0	&	0	&	-1	
	\end{bmatrix} 
\end{equation*}

Now, we calculate the joint state $\ket{\psi_{AB}}$ as follows
\begin{align*}
	\ket{\psi_{AB}} &= \ket{\psi_A} \otimes \ket{\psi_B} \\
	&= \begin{bmatrix}
		\cos(\theta_A)\cos(\theta_B) \\
		- \cos(\theta_A)\sin(\theta_B) \\
		\sin(\theta_A)\cos(\theta_B) \\
		- \sin(\theta_A)\sin(\theta_B)
	\end{bmatrix}
\end{align*}

Now, we use the above expressions for $C$ and $\ket{\psi_{AB}}$ to calculate the expectation $\langle C \rangle$
\begin{align*}
	\langle C \rangle &= \bra{\psi_{AB}} C \ket{\psi_{AB}}	\\
	&= -\sqrt{2}
	\begin{bmatrix}
		\cos(\theta_A)\cos(\theta_B) \\
		- \cos(\theta_A)\sin(\theta_B) \\
		\sin(\theta_A)\cos(\theta_B) \\
		- \sin(\theta_A)\sin(\theta_B)
	\end{bmatrix}^T
	\begin{bmatrix}
		-1	&	0	&	0	&	-1	\\
		0	&	1	&	-1	&	0	\\
		0	&	-1	&	1	&	0	\\
		-1	&	0	&	0	&	-1	
	\end{bmatrix} 
	\begin{bmatrix}
		\cos(\theta_A)\cos(\theta_B) \\
		- \cos(\theta_A)\sin(\theta_B) \\
		\sin(\theta_A)\cos(\theta_B) \\
		- \sin(\theta_A)\sin(\theta_B)
	\end{bmatrix} \\
	\begin{split}
	\;=\sqrt{2}(
		-\cos^2(\theta_A)\cos^2(\theta_B)
		+ \cos(\theta_A)\cos(\theta_B)\sin(\theta_A)\sin(\theta_B) \\
		+ \cos^2(\theta_A)\sin^2(\theta_B)
		+ \cos(\theta_A)\sin(\theta_B)\sin(\theta_A)\cos(\theta_B) \\
		+ \sin(\theta_A)\cos(\theta_B)\cos(\theta_A)\sin(\theta_B)
		+ \sin^2(\theta_A)\cos^2(\theta_B)\\
		+ \sin(\theta_A)\sin(\theta_B)\cos(\theta_A)\cos(\theta_B)
		- \sin^2(\theta_A)\sin^2(\theta_B)
	)
	\end{split} \\
	&= -\sqrt{2}\cos(2\theta_A + 2\theta_Y)
\end{align*}

Thus, one can see that the maximum possible value for $\langle C \rangle$ is $\sqrt{2}$, which occurs when $(2\theta_A + 2\theta_Y = 1)$.

