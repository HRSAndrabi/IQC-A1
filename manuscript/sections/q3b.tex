\textbf{Problem 3b: Grover's algorithm (single iteration)}. Use the oracle which you constructed in part (a) to implement a single iteration of Grover's algorithm, starting from an equal superposition, and calculate the probability of measuring a number which is in the given set of numbers at the output. 
Show your working.


\textbf{Answer}. Continue with the same notation as part (a), with $S = \{1, 3, 7, 22, 25\}$ denoting the set of solutions and
$M=|S|$ denoting the cardinality of $S$.
We begin with an initial state in equal superposition
\begin{align*}
\ket{\phi} &= \frac{1}{\sqrt{N}}\sum_{i=0}^{N-1}\ket{i}
\end{align*}

We can rewrite this state using a convenient two-dimensional basis of solutions, $\ket{a}$, and non-solutions, $\ket{b}$
\begin{align*}
	\ket{a} &= \frac{1}{\sqrt{M}}\sum_{i \in S}\ket{i},\;\;\; \ket{b} = \frac{1}{\sqrt{N-M}}\sum_{i \notin S}\ket{i} \\
	\ket{\Phi_0} &= \alpha \ket{a} + \beta \ket{b},\;\; \text{where} \\ 
	\alpha &= \frac{\sqrt{M}}{\sqrt{N}}, \;\;\; \beta = \frac{\sqrt{N-M}}{\sqrt{N}}
\end{align*}

In this basis, the initial state $\ket{\Phi_0}$ forms an angle $\theta = \arcsin (\frac{\sqrt{M}}{\sqrt{N}})$ with the horizontal axis, and each iteration of Grover's algorithm increases this angle by exactly $2\theta$.
After one iteration of Grover's algorithm, the initial state $\ket{\Phi_0} = (1, \theta)$ in polar form is modified to $\ket{\Phi_1} = (1, 3\theta)$.

One can calculate the probability of measuring a solution $S^\prime \in S$ by projecting $\ket{\Phi_1}$ on to the vertical axis
\begin{align*}
	Pr(S^\prime) &= \arcsin(3\theta) \\
	&= \arcsin(\frac{3\sqrt{5}}{\sqrt{32}}) \\
	&\approx 0.9388
\end{align*}


