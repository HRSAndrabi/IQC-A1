\textbf{Problem 3a: Oracle construction}. Construct an oracle with five qubits, which identifies (i.e. applies a phase to) numbers in the set $(1, 3, 7, 22, 25)$. 
You may use multiply controlled operations.
Optimise your circuit to use as few operations as possible. 
Briefly describe your implementation of this circuit including any optimisations which you have made.


\textbf{Answer}. Let $S = \{1, 3, 7, 22, 25\}$ denote the set of solutions. 
Let $M=|S|$ denote the cardinality of $S$.
We can construct an oracle to apply a phase to the computational states $\ket{i} \forall i \in S$ using a series of $X$ gates and controlled-$Z$ gates. The general intuition of an example (non-optimised) circuit is as follows: 

Given a target $S^\prime \in S$, we first apply $X$ gates to flip all qubits corresponding to $0$-valued bits in the binary representation of $S^\prime$. 
Then, we apply a controlled-$Z$ gate (controlled on $\ket{1}$ state) with any given qubit as a target, and all remaining qubits as controls.
This controlled-$Z$ gate flips the phase of the target qubit if the input state $\psi_0$ matches the binary representation of $S^\prime$.
We then revert each qubit back to it's initial computational state using $X$ gates.
In this way, we construct a complete circuit by chaining $M$ components, where the action of each component is such that it applies a phase to a unique computational state $S^\prime \in S$.

One can optimise this circuit to use only controlled-$Z$ gates, by methodically setting the control computational states to match the binary representations of solutions in $S$.
For example, one could detect an input string $00001$ $(S^\prime=1)$ using a CNOT gate that is controlled by the $\ket{0}$ state in the first four qubits, and the $\ket{1}$ state in the final qubit.
This optimised circuit\footnote{Access this circuit at \url{https://qui.science.unimelb.edu.au/circuits/642e4c77d5f0500054592227
}} achieves the target function of the Oracle in five operations.

