\textbf{Problem 5a: CHSH inequality (classical states)}. Suppose that $Q, R, S$, and $T$ take values $\{\pm 1\}$. 
Consider the quantity composed of combined operators: $QS + RS + RT - QT$. 
Note that there is an implied tensor product in these combined operators.
In this case, what is the upper bound on the expectation (mean value) of the quantity $QS + RS + RT - QT$? 
This is called Bell's inequality.

\textbf{Answer}. Since $S, T \in \{-1, 1\}$, we have either $S=T$ or $S=-T$.
Now consider the following combination
\begin{equation}
	\label{eqn:CHSH}
	QS + RS + RT - QT = Q(S - T) + R(S + T)
\end{equation}

In the case $S=T$, $Q(S - T) =0$, while in the case $S=-T$, $R(S + T) =0$.
In either eventuality, one of the terms on the RHS of equation \ref{eqn:CHSH} will become zero, and the remaining term will take the values $\pm2$.
Thus, if the measurement is repeated over $n$ independent trials, the upper bound on the expectation of the measured quantity is given by
\begin{equation*}
	\frac{1}{n}\sum_{i=1}^n (Q_iS_i + R_iS_i + R_iT_i - Q_iT_i) \leq 2
\end{equation*}




