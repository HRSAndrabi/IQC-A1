\textbf{Problem 6a: Quantum circuit (3-qubit)}. Design a quantum circuit that implements $f(x)$ for $x$ represented by $n=3$ qubits (in ket notation, define the left-most qubit corresponding to the most significant bit). 
Draw the quantum circuit using one-qubit gates, two-qubit gates and as few Toffoli gates as possible. 
Explain its working. 
You may use ancilla qubits.


\textbf{Answer}.\footnote{Access the circuit at \url{https://qui.science.unimelb.edu.au/circuits/642d7e3895866d00129d399c}} Using three input qubits $(\ket{x_1},\ket{x_2},\ket{x_3})$ and one output qubit $\ket{y}$, we can design a simple quantum circuit to implement $f(x)$ using three Toffoli gates. 
\begin{center}
	\begin{quantikz}[column sep=3cm]
		\lstick{$\ket{x_1}$} & \ctrl{3} & \qw & \ctrl{3} & \qw \\
		\lstick{$\ket{x_2}$} & \control{} & \ctrl{2} & \qw & \qw \\
		\lstick{$\ket{x_3}$} & \qw & \control{} & \control{} & \qw \\
		\lstick{$\ket{y}$} & \targ{} & \targ{} & \targ{} & \qw \\
	\end{quantikz}
\end{center}

Each Toffoli gate monitors a unique two-bit pair, and flips the output qubit ${\ket{y}}$ if both bits in the target pair are set to one.
In the event that all three bits are set to one, an odd number of Toffoli gates ensures that the output qubit registers $1$.

Note: each Toffoli gate can, in principle, be reproduced using single-qubit gates alone, as in problem (2).
As such, this circuit could be reconstructed without explicit use of any Toffoli gates. 
For brevity of notation, this exercise is avoided.

